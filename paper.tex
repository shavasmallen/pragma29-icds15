\documentclass{acm_proc_article-sp}

\usepackage{url}

\begin{document}

\title{Experiences designing and building a PRAGMA Cloud Scheduler}
\subtitle{[Short Paper]}

\numberofauthors{3} 

\author{
% 1st author      
\alignauthor
Shava Smallen\\
      \affaddr{San Diego Supercomputer Center}\\
      \affaddr{University of California San Diego}\\
       \email{ssmallen@sdsc.edu}
% 2nd. author
\alignauthor
Nadya Williams\\
      \affaddr{San Diego Supercomputer Center}\\
      \affaddr{University of California San Diego}\\
       \email{nadya@sdsc.edu}
\and % 3rd. author
\alignauthor Philip Papadopoulos \\
      \affaddr{San Diego Supercomputer Center}\\
      \affaddr{University of California San Diego}\\
       \email{phil@sdsc.edu}
%\and  % use '\and' if you need 'another row' of author names
}

\maketitle
\begin{abstract}
The Pacific Rim Application and Grid Middleware Assembly (PRAGMA) is a community of individuals and institutions from around the Pacific Rim that actively collaborate to enable scientific expeditions in areas like biodiversity and lake ecology.  Over the past X years, the technology focus for PRAGMA partners has shifted to cloud and software defined networking as enabling technologies.  During PRAGMA 27, the Resources Working group discussed rebooting the persistent PRAGMA Testbed as a way for institutions to contribute and use shared resources, leveraging technologies such as PRAGMA Boot, Personal Cloud Controller (PCC), and overlay networks.  To make PRAGMA resource sharing easier, a lightweight scheduler was proposed and discussed as a way to enable access to the resources and to manage resource reservations.  This short paper discusses the design process and initial prototype of a simple cloud scheduler for PRAGMA.  

%add contribute and leverage shared resources
\end{abstract}

% A category with the (minimum) three required fields
%\category{H.4}{Information Systems Applications}{Miscellaneous}
%A category including the fourth, optional field follows...
%\category{D.2.8}{Software Engineering}{Metrics}[complexity measures, performance measures]

%terms{Cloud, Scheduling, Resource Sharing}

\keywords{Cloud, Scheduling, Resource Sharing} % NOT required for Proceedings

\section{Introduction}

some text~\cite{pragmaWeb}

Repeat longer version of text in abstract with appropriate citations.

Discuss requirements and technologies like pragma boot, etc.

% enable varying -- time limited -- not get bogged down by deployment of technologies

% view available resources and scientific application virtual machines
\section{Scheduler Design}

Discuss process of narrowing down scheduler design from meeting ideas 
(e.g., batch scheduler, Grid 5000, GENI, Google doc, DHCP leases).

\section{Calendaring systems}

Discuss selection of Booked.

\section{Cloud Scheduler Pilot}

Discuss creation of Cloud Scheduler Pilot.

\section{Conclusions and Future Work}

Some interesting conclusion.

%ACKNOWLEDGMENTS are optional
%\section{Acknowledgments}

\bibliographystyle{abbrv}
\bibliography{paper}  

\end{document}

